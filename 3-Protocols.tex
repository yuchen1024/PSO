\section{Commutative Pseudorandom Functions}
Let $F$ be a family of pseudorandom functions from $D$ to $R$. The standard security requirement for PRFs is pseudorandomness. 

\begin{trivlist}
\item \textbf{Pseudorandomness.} Let $\mathcal{A}$ be an adversary against PRFs and define its advantage as:
\begin{displaymath}
    \AdvA =
        \Pr \left[ b = b':~
        \begin{array}{l}
            pp \leftarrow \mathsf{Setup}(1^\lambda);\\
            k \leftarrow \mathsf{KeyGen}(pp);\\
            b \leftarrow \{0,1\}; \\
            b' \leftarrow \mathcal{A}^{\mathcal{O}_\mathsf{ror}(b, \cdot)}(\lambda);
        \end{array} 
        \right] - \frac{1}{2},
\end{displaymath}
where $\mathcal{O}_\mathsf{ror}(0, x) = F_k(x)$, $\mathcal{O}_\mathsf{ror}(1, x) = \mathsf{H}(x)$ 
(here $\mathsf{H}$ is chosen uniformly at random from all the functions 
from $D$ to $R$\footnote{To efficiently simulate access to a uniformly random function $\mathsf{H}$ from $D$ to $R$, 
one may think of a process in which the adversary's queries to $\mathcal{O}_\mathsf{ror}(1, \cdot)$ 
are ``lazily'' answered with independently and randomly chosen elements in $R$, 
while keeping track of the answers so that queries made repeatedly are answered consistently.}). 
Note that $\mathcal{A}$ can adaptively access the oracle $\mathcal{O}_\mathsf{ror}(b, \cdot)$ polynomial many times.
We say that $F$ is pseudorandom if for any PPT adversary its advantage function 
$\AdvA$ is negligible in $\lambda$. We refer to such security as full PRF security. 

Sometimes the full PRF security is not needed and it is sufficient 
if the function cannot be distinguished from a uniform random one when challenged on random inputs. 
The formalization of such relaxed requirement is \emph{weak pseudorandomness}, 
which is defined the same way as pseudorandomness except that the inputs of 
oracle $\mathcal{O}_\mathsf{ror}(b, \cdot)$ are uniformly chosen from $D$ by the challenger
instead of adversarially chosen by $\mathcal{A}$. 
PRFs that satisfy weak pseudorandomness are referred to as \emph{weak PRFs}.     
\end{trivlist}

\begin{trivlist}
\item \textbf{Composable.} For a family of keyed function $F$, $F$ is 2-composable if $R \subseteq D$, 
    namely, for any $k_1, k_2 \in K$, the function $F_{k_1}(F_{k_2}(\cdot))$ is well-defined. 
    From now on, we will assume $R = D$ for simplicity.   

\item \textbf{Commutative.} For a family of composable keyed function, we say it is commutative if:  
\begin{gather*}
    \forall k_1, k_2 \in K, \forall x \in X: F_{k_1}(F_{k_2}(x)) = F_{k_2}(F_{k_1}(x))
\end{gather*}
\end{trivlist}

It is easy to see that the standard pseudorandomness denies commutative property, 
but weak pseudorandomness and commutative property may co-exist.

\subsection{Commutative Weak PRFs from the DDH Assumption}
We build a concrete commutative weak PRF from the DDH assumption. 
\begin{itemize}
\item $\mathsf{Setup}(1^\lambda)$: runs $\mathsf{GroupGen}(1^\lambda) \rightarrow (\mathbb{G}, g, p)$, 
    outputs $pp = (\mathbb{G}, g, p)$. 

\item $\mathsf{KeyGen}(pp)$: outputs $k \sample \mathbb{Z}_p$. 
    Each $k$ defines a function from $\mathbb{G}$ to $\mathbb{G}$, 
    which takes $x \in \mathbb{G}$ as input and outputs $x^k$. 
\end{itemize}
It is straightforward to verify that $F$ is commutative. 
By the random self-reducibility of the DDH assumption, 
the weak pseudorandomness can be tightly reduced to the DDH assumption. 
We omit the details for triviality. 

\section{PSU from Commutative Weak PRFs}
\begin{figure}[!hbtp]
\begin{framed}
\begin{minipage}[center]{\textwidth}
\begin{trivlist}
\item \textbf{Parameters:} 
\begin{itemize}
    \item Common input: $F: K \times D \rightarrow D$, hash function $\mathsf{H}: \{0,1\}^* \rightarrow D$. 

    \item Input of sender $\mathcal{S}$: $X = \{x_1, \dots, x_n\}$.

    \item Input of receiver $\mathcal{R}$: $Y = \{y_1, \dots, y_n\}$. 
\end{itemize}

\item \textbf{Protocol:}

\begin{enumerate}
\item $\mathcal{R}$ picks $k_2 \sample K$, 
    then sends $\{F_{k_2}(\mathsf{H}(y_1)), \dots, F_{k_2}(\mathsf{H}(y_n))\}$ to $\mathcal{S}$. 

\item $\mathcal{S}$ picks $k_1 \sample K$, 
    computes and sends $\{F_{k_1}(\mathsf{H}(x_1)), \dots, F_{k_1}(\mathsf{H}(x_n))\}$ to $\mathcal{R}$; 
    then computes $\{F_{k_1}(F_{k_2}(\mathsf{H}(y_1)), \dots, F_{k_1}(F_{k_2}(\mathsf{H}(y_n))\}$, 
    sends its permutation $\Gamma$ to $\mathcal{R}$; 

\item $\mathcal{R}$ computes $\{F_{k_2}(F_{k_1}(\mathsf{H}(x_1)), \dots, F_{k_2}(F_{k_1}(\mathsf{H}(x_n))\}$, 
    then sets $v_i = 0$ iff the value is not in $\Gamma$. 

\item $\mathcal{R}$ with select vector $(v_1, \dots, v_n)$ and $\mathcal{S}$ 
    with input $\{(x_i, \bot)\}_{i \in [n]}$ engage in one-sided OT. 

\item $\mathcal{R}$ obtains $X - X \cap Y$, and thus obtains the union $X \cup Y$.  
\end{enumerate}
\end{trivlist}
\end{minipage}
\end{framed}
\caption{PSU from Commutative weak PRFs}\label{fig:PSU-from-commutative-functions}
\end{figure} 

\begin{trivlist}
\item \textbf{Correctness.} The above protocol is correct except the case $E$ that 
    $F_{k_1}(F_{k_2}(\mathsf{H}(x))) = F_{k_1}(F_{k_2}(\mathsf{H}(y)))$ for some $x \neq y$ occurs. 
    We further divide $E$ to $E_0$ and $E_1$. 
    $E_0$ denotes the case that $\mathsf{H}(x) = \mathsf{H}(y)$. 
    $E_1$ denotes the case that $\mathsf{H}(x) \neq \mathsf{H}(y)$ 
    but $F_{k_1}(F_{k_2}(\mathsf{H}(x))) = F_{k_1}(F_{k_2}(\mathsf{H}(y)))$, 
    which can further be divided into sub-cases $E_{10}$ — $F_{k_2}(\mathsf{H}(x)) = F_{k_2}(\mathsf{H}(y))$ 
    and $E_{11}$ — $F_{k_2}(\mathsf{H}(x)) \neq F_{k_2}(\mathsf{H}(y))$ but 
    $F_{k_1}(F_{k_2}(\mathsf{H}(x))) = F_{k_1}(F_{k_2}(\mathsf{H}(y)))$. 
    By the collision resistance of $\mathsf{H}$, we have $\Pr[E_0] = 2^{-\sigma}$. 
    By the weak pseudorandomness of $F$, we have $\Pr[E_{10}] = \Pr[E_{11}] = 2^{-\ell}$. 
    Therefore, we have $\Pr[E] \leq \Pr[E_0] + \Pr[E_{10}] + \Pr[E_{11}] = 2^{-\sigma}+2^{-\ell+1}$. 

\end{trivlist} 

\begin{theorem}
The above PSU protocol is secure in the semi-honest model assuming $\mathsf{H}$ is a random oracle 
and $F$ is a family of commutative weak PRFs.
\end{theorem}

\begin{proof}
We exhibit simulators $\SimR$ and $\SimS$ for simulating corrupt $\mathcal{R}$ and $\mathcal{S}$ respectively, 
and argue the indistinguishability of the produced transcript from the real execution. Let $|X \cap Y| = m$. 

\begin{trivlist}
\item \underline{Corrupt sender:} $\SimS$ simulates the view of corrupt $\mathcal{S}$, 
    which consists of $\mathcal{S}$'s randomness, input, output and received messages.
 
    We argue the output of $\SimS$ is indistinguishable from the real execution. 
    For this, we formally show the simulation by proceeding the sequence of hybrid transcripts, 
    where $T_0$ is the real view of $\mathcal{S}$, and $T_2$ is the output of $\SimS$. 

\item $\text{Hybrid}_0$: $\SimS$ simulates with the knowledge of $Y$. 
\begin{itemize}
    \item $\SimS$ chooses the randomness for $\mathcal{R}$, i.e., picks $k_2 \sample K$.
    
    \item RO queries: for random oracle query $\langle z \rangle$, 
        picks $h \sample D$ and set $\mathsf{H}(z):=h$. 

    \item Let $h_i = \mathsf{H}(y_i)$ for each $y_i \in Y$. $\SimS$ outputs $(F_{k_2}(h_1), \dots, F_{k_2}(h_n))$.     
\end{itemize}
Clearly, $\SimS$'s simulation is identical to the real view. 

\begin{center}
\begin{tikzpicture}
\node[textnode, name = intersection] {$X \cap Y$}; 
\node[ellipsenode, name = X, left of =intersection, xshift=-1.5em, 
    minimum height = 3em, minimum width=6em, label={[xshift=-1em]$X$}] {};
\node[ellipsenode, name = Y, left of =intersection, xshift=1.5em, 
    minimum height = 3em, minimum width=6em, label={[xshift=1em]$Y$}] {}; 

\node[textnode, right of = Y, xshift=10em] {$\mathsf{H}(z_i): = h_i \sample D$};
\end{tikzpicture}
\end{center}

\item $\text{Hybrid}_1$: $\SimS$ simulates without the knowledge of $Y$, 
    and changes the simulation in the final input.  
\begin{itemize}
    \item $\SimS$ outputs $(s_1, \dots, s_n)$ where $s_i \sample D$.       
\end{itemize}
We argue that the view in hybrid 0 and hybrid 1 are computationally indistinguishable. 
More precisely, a PPT adversary $\mathcal{A}$ \red{(with knowledge of $X$ and $Y$)} 
against commutative weak PRF (with secret key $k$) is given $n$ tuples $(h_i, s_i)$ where $h_i \sample D$, 
and is asked to distinguish if $s_i = F_k(h_i)$ or $s_i$ are random values. 
$\mathcal{A}$ implicitly sets $\mathcal{R}$'s randomness $k_2:=k$. 
\begin{itemize}
    \item RO queries: for random oracle query $\langle z \rangle$ where $z \notin Y$,  
        picks $h \sample D$ and set $\mathsf{H}(z):=h$; for random oracle query $\langle y_i \rangle$ where $y_i \in Y$, 
        sets $\mathsf{H}(y_i):=h_i$.  

    \item $\mathcal{A}$ outputs $(s_1, \dots, s_n)$.      
\end{itemize}

If $s_i = F_k(h_i)$ for $i \in [n]$, then $\mathcal{A}$'s simulation is identical to hybrid 0. 
If $s_i$ are random values, then $\mathcal{A}$'s simulation is identical to hybrid 1. 
\end{trivlist}

\begin{trivlist}
\item \underline{Corrupt receiver:} $\SimR$ simulates the view of corrupt $\mathcal{R}$, 
    which consists of $\mathcal{R}$'s randomness, input, output and received messages.
 
    We argue the output of $\SimR$ is indistinguishable from the real execution. 
    For this, we formally show the simulation by proceeding the sequence of hybrid transcripts, 
    where $T_0$ is the real view of $\mathcal{R}$, and $T_2$ is the output of $\SimR$. 

\item $\text{Hybrid}_0$: $\SimR$ simulates with the knowledge of $X$. 
\begin{itemize}
    \item $\SimR$ chooses the randomness for $\mathcal{S}$, i.e., picks $k_1 \sample K$.
    
    \item RO queries: for random oracle query $\langle z \rangle$, picks $h \sample D$ and sets $\mathsf{H}(z):=h$. 

    \item $\SimR$ computes and outputs 
        $\{F_{k_1}(\mathsf{H}(x_i))\}_{x_i \in X}$, 
        computes $\{F_{k_1}(F_{k_2}(\mathsf{H}(y_i))\}_{y_i \in Y}$, outputs its permutation $\Gamma$.      
\end{itemize}
Clearly, $\SimR$'s simulation is identical to the real view. 

\begin{center}
\begin{tikzpicture}
\node[textnode, name = intersection] {$X \cap Y$}; 
\node[ellipsenode, name = X, left of =intersection, xshift=-1.5em, 
    minimum height = 3em, minimum width=6em, label={[xshift=-1em]$X$}] {};
\node[ellipsenode, name = Y, left of =intersection, xshift=1.5em, 
    minimum height = 3em, minimum width=6em, label={[xshift=1em]$Y$}] {}; 

\node[textnode, right of = Y, xshift=10em] {$\mathsf{H}(z_i): = h_i \sample D$};
\end{tikzpicture}
\end{center}

\item $\text{Hybrid}_1$: $\SimR$ simulates without the knowledge of $X \cap Y$, 
    but with the knowledge of $X - X \cap Y$.   
\begin{itemize}
    \item $\SimR$ picks $s_j \sample D$ for $j \in [n-m]$ (associated with PRF values for elements in $X - X \cap Y$), 
        picks $s_k \sample D$ for $k \in [m]$ (implicitly associated with PRF values for elements in $X \cap Y$, 
        though the exact elememts are unknown), 
        outputs $(\{s_j\}_{j \in [n-m]}, \{s_k\}_{k \in m})$ according to the order of the final selection vector; 
        picks $s_\ell \sample D$ for $\ell \in [n-m]$ (associated with PRF values for elements in  $Y - X \cap Y$),  
        outputs the permutation of $(\{F_{k_2}(s_k)\}_\{k \in [m]\}, \{F_{k_2}(s_\ell)\}_{\ell \in [n-m]})$.                
\end{itemize}

We argue that the view in hybrid 1 and hybrid 2 are computationally indistinguishable. 
More precisely, a PPT adversary $\mathcal{A}$ \red{(with knowledge of $X$ and $Y$)} against commutative weak PRFs 
are given $n+m$ tuples $(h_i, s_i)$ where $h_i \sample D$, and is asked to determine if $s_i = F_{k}(h_i)$ or random values. 
$\mathcal{A}$ implicitly sets $\mathcal{S}$'s randomness $k_1:=k$, picks $k_2 \sample K$. 
\begin{itemize}
    \item RO queries: for $z \notin X \cup Y$, picks $h \sample D$ and returns $\mathsf{H}(z):=h$; 
        for $z_i \in X \cup Y$, returns $\mathsf{H}(z_i) = h_i$. 

    \item $\mathcal{A}$ prepares $s_j$ for $z_i \in X - X \cap Y$, 
        $s_k$ for $z_k \in X \cap Y$, 
        outputs $(\{s_j\}_{j \in [n-m]}, \{s_k\}_{k \in m})$ according to the order of the final selection vector; 
        prepares $s_\ell \sample D$ for $z_\ell \in Y - X \cap Y$,  
        outputs the permutation of $(\{F_{k_2}(s_k)\}_{k \in [m]}, \{F_{k_2}(s_\ell)\}_{\ell \in [n-m]})$.               
\end{itemize}

If $s_i$ are function values, then the simulation is identical to hybrid 0, 
else it is identical to hybrid 1. 
\end{trivlist}
\end{proof}






\section{Instantiation Based on the DDH Assumption}\label{sec:protocols}
We construct a linear complexity PSU protocol from the DDH assumption in Figure~\ref{fig:DH-PSU}.  
\begin{figure}[!hbtp]
\begin{framed}
\begin{minipage}[center]{\textwidth}
\begin{trivlist}
\item \textbf{Parameters:} 
\begin{itemize}
    \item Common input: $\mathbb{G} = \langle g \rangle$, hash function $\mathsf{H}: \{0,1\}^* \rightarrow \mathbb{G}$. 

    \item Input of sender $\mathcal{S}$: $X = \{x_1, \dots, x_n\}$.

    \item Input of receiver $\mathcal{R}$: $Y = \{y_1, \dots, y_n\}$. 
\end{itemize}

\item \textbf{Protocol:}

\begin{enumerate}
\item $\mathcal{R}$ picks $b \sample \mathbb{Z}_p$, then sends $(\mathsf{H}(y_1)^b, \dots, \mathsf{H}(y_n)^b)$ to $\mathcal{S}$. 

\item $\mathcal{S}$ picks $a \sample \mathbb{Z}_p$, 
    sends $(\mathsf{H}(x_1)^a, \dots, \mathsf{H}(x_n)^a)$ to $\mathcal{R}$; 
    he also computes $(\mathsf{H}(y_1)^b)^a, \dots, \mathsf{H}(y_n)^b)^a)$ and sends its permutation $\Gamma$ to $\mathcal{R}$. 


\item $\mathcal{R}$ computes $(\mathsf{H}(x_1)^a)^b, \dots, \mathsf{H}(x_n)^a)^b)$, 
    then sets $v_i = 0$ iff the value is not in $\Gamma$. 

\item $\mathcal{R}$ with select vector $(v_1, \dots, v_n)$ and $\mathcal{S}$ with input $\{(x_i, \bot)\}_{i \in [n]}$ 
    engage in one-sided OT. 

\item $\mathcal{R}$ obtains $X - X \cap Y$, and thus obtains the union $X \cup Y$.  
\end{enumerate}
\end{trivlist}
\end{minipage}
\end{framed}
\caption{DH-PSU}\label{fig:DH-PSU}
\end{figure}

\begin{theorem}
The above PSU protocol is secure in the semi-honest model assuming $\mathsf{H}$ is a random oracle and the DDH assumption.
\end{theorem}

\begin{proof}
We exhibit simulators $\SimR$ and $\SimS$ for simulating corrupt $\mathcal{R}$ and $\mathcal{S}$ respectively, 
and argue the indistinguishability of the produced transcript from the real execution. Let $|X \cap Y| = m$. 


\begin{trivlist}
\item \underline{Corrupt sender:} $\SimS$ simulates the view of corrupt $\mathcal{S}$, 
    which consists of $\mathcal{S}$'s randomness, input, output and received messages.

    Intuitively, the crux of the security proof is to simulate sender's view without the knowledge of $Y$, 
    more precisely, the knowledge of $X \cap Y$. 
    Looking ahead, we will use RO to program the hash outputs for elements in $Y - X \cap Y$ naively, 
    and program the hash outputs for elements in $X$ in a special way to ensure the associated messages sent from $\mathcal{R}$ 
    are pseudorandom. 
 
    We argue the output of $\SimS$ is indistinguishable from the real execution. 
    For this, we formally show the simulation by proceeding the sequence of hybrid transcripts, 
    where $T_0$ is the real view of $\mathcal{S}$, and $T_2$ is the output of $\SimS$. 

\item $\text{Hybrid}_0$: $\SimS$ simulates with the knowledge of $Y$. 
\begin{itemize}
    \item $\SimS$ chooses the randomness for $\mathcal{R}$, i.e., picks $b \sample \mathbb{Z}_p$.
    
    \item RO queries: picks $h_i \sample \mathbb{G}$ and sets $\mathsf{H}(z_i):=h_i$. 

    \item $\SimS$ outputs $h_i^b$ for $y_i \in Y$.     
\end{itemize}
Clearly, $\SimS$'s simulation is identical to the real view. 

\begin{center}
\begin{tikzpicture}
\node[textnode, name = intersection] {$X \cap Y$}; 
\node[ellipsenode, name = X, left of =intersection, xshift=-1.5em, 
    minimum height = 3em, minimum width=6em, label={[xshift=-1em]$X$}] {};
\node[ellipsenode, name = Y, left of =intersection, xshift=1.5em, 
    minimum height = 3em, minimum width=6em, label={[xshift=1em]$Y$}] {}; 

\node[textnode, right of = Y, xshift=10em] {$\mathsf{H}(z_i): = h_i \sample \mathbb{G}$};
\end{tikzpicture}
\end{center}

\item $\text{Hybrid}_1$: $\SimS$ still simulates with the knowledge of $Y$, 
    but slightly change the simulation of random oracle 
\begin{itemize}
    \item RO queries: picks $d_i, e_i \sample \mathbb{Z}_p$, sets $\mathsf{H}(z_i):= h_i = h^{d_i} \cdot g^{e_i}$. 
        Here, $h$ is a fixed random group element.  

    \item $\SimS$ outputs $h_i^b$ for $y_i \in Y$.            
\end{itemize}
The modification in RO simulation does not alter the view. 

\begin{center}
\begin{tikzpicture}
\node[textnode, name = intersection] {$X \cap Y$}; 
\node[ellipsenode, name = X, left of =intersection, xshift=-1.5em, 
    minimum height = 3em, minimum width=6em, label={[xshift=-1em]$X$}] {};
\node[ellipsenode, name = Y, left of =intersection, xshift=1.5em, 
    minimum height = 3em, minimum width=6em, label={[xshift=1em]$Y$}] {}; 

\node[textnode, right of = Y, xshift=10em] {$\mathsf{H}(z_i) := h_i = h^{d_i} \cdot g^{d_i}$}; 
\end{tikzpicture}
\end{center}

\item $\text{Hybrid}_2$: $\SimS$ simulates without the knowledge of $Y$, 
    and changes the simulation in the final input.  
\begin{itemize}
    \item $\SimS$ outputs random $f_i$ for $i \in [n]$.       
\end{itemize}
We argue that the view in hybrid 2 and hybrid 3 are computationally indistinguishable. 
More precisely, given $(g, g^\alpha, g^\beta, g^\gamma)$, 
a PPT adversary $\mathcal{A}$ \red{(with knowledge of $X$ and $Y$)} implicitly sets $\mathcal{R}$'s randomness $b:=\alpha$. 
\begin{itemize}
    \item RO queries: sets $h := g^\beta$; picks $d_i, e_i \sample \mathbb{Z}_p$, sets $\mathsf{H}(z_i):=h^{d_i} \cdot g^{e_i}$. 

    \item $\mathcal{A}$ outputs $g^{\gamma d_i + \alpha e_i}$ for $i \in [n]$.      
\end{itemize}

If $(g, g^\alpha, g^\beta, g^\gamma)$ is a DDH tuple, then the simulation is identical to hybrid 1, 
else it is identical to hybrid 2. 
\end{trivlist}

\begin{trivlist}
\item \underline{Corrupt receiver:} $\SimR$ simulates the view of corrupt $\mathcal{R}$, 
    which consists of $\mathcal{R}$'s randomness, input, output and received messages.
 
    We argue the output of $\SimR$ is indistinguishable from the real execution. 
    For this, we formally show the simulation by proceeding the sequence of hybrid transcripts, 
    where $T_0$ is the real view of $\mathcal{R}$, and $T_2$ is the output of $\SimR$. 

\item $\text{Hybrid}_0$: $\SimR$ simulates with the knowledge of $X$. 
\begin{itemize}
    \item $\SimR$ chooses the randomness for $\mathcal{S}$, i.e., picks $a \sample \mathbb{Z}_p$.
    
    \item RO queries: picks $h_i \sample \mathbb{G}$ and sets $\mathsf{H}(z_i):=h_i$. 

    \item $\SimR$ computes and outputs $\{h_i^{a}\}_{x_i \in X}$; 
        computes $\{(h_i^{b})^a\}_{y_i \in Y}$, outputs its permutation.      
\end{itemize}
Clearly, $\SimR$'s simulation is identical to the real view. 

\begin{center}
\begin{tikzpicture}
\node[textnode, name = intersection] {$X \cap Y$}; 
\node[ellipsenode, name = X, left of =intersection, xshift=-1.5em, 
    minimum height = 3em, minimum width=6em, label={[xshift=-1em]$X$}] {};
\node[ellipsenode, name = Y, left of =intersection, xshift=1.5em, 
    minimum height = 3em, minimum width=6em, label={[xshift=1em]$Y$}] {}; 

\node[textnode, right of = Y, xshift=10em] {$\mathsf{H}(z_i): = h_i \sample \mathbb{G}$};
\end{tikzpicture}
\end{center}

\item $\text{Hybrid}_1$: $\SimR$ still simulates with the knowledge of $X$, 
    but slightly changes the simulation of random oracle 
\begin{itemize}
    \item RO queries: for $z_i \notin Y$, picks $r_i \sample \mathbb{Z}_p$ and sets $\mathsf{H}(z_i):= h_i = g^{r_i}$; 
        for $z_i \in Y$, 
        picks $d_i, e_i \sample \mathbb{Z}_p$, sets $\mathsf{H}(z_i):= h_i = h^{d_i} \cdot g^{e_i}$. 
        Here, $h$ is a fixed random group element.  

    \item $\SimR$ computes and outputs $\{h_i^{a}\}_{x_i \in X}$; 
        computes $\{(h_i^{b})^a\}_{y_i \in Y}$, outputs its permutation.               
\end{itemize}
The modification in RO simulation does not alter the view. 

\begin{center}
\begin{tikzpicture}
\node[textnode, name = intersection] {$X \cap Y$}; 
\node[ellipsenode, name = X, left of =intersection, xshift=-1.5em, 
    minimum height = 3em, minimum width=6em, label={[xshift=-1em]$X$}] {};
\node[ellipsenode, name = Y, left of =intersection, xshift=1.5em, 
    minimum height = 3em, minimum width=6em, label={[xshift=1em]$Y$}] {}; 

\node[textnode, right of = Y, xshift=14em] {\(\displaystyle
\mathsf{H}(z_i) = \left\{ \begin{array}{cl}
g^{r_i} & \text{if } z_i \notin Y\\
h^{d_i} \cdot g^{d_i}  & \text{if } z_i \in Y\\ 
\end{array} \right. \)};  
\end{tikzpicture}
\end{center}

\item $\text{Hybrid}_2$: $\SimR$ simulates without the knowledge of $X$, 
    and changes the simulation in the final input.  
\begin{itemize}
    \item $\SimR$ computes $f_j = h_j^{a}$ for $x_j \in X - X \cap Y$ (note that the information of such $x_j$ 
        can be inferred from $\mathcal{R}$'s output), picks random $f_k$ for $k \in [m]$ 
        (implicitly assigned for elements in $X \cap Y$), outputs $(f_1, \dots, f_n)$; 
        then picks random $f_\ell$ for $\ell \in [n-m]$ (implicitly assigned to elements in $Y - X \cap Y$), 
        outputs the permutation of $(\{f_k^b\}_{k \in [m]}, \{f_\ell^b\}_{\ell \in [n-m]})$.        
\end{itemize}
We argue that the view in hybrid 2 and hybrid 3 are computationally indistinguishable. 
More precisely, given $(g, g^\alpha, g^\beta, g^\gamma)$, 
a PPT adversary $\mathcal{A}$ \red{(with knowledge of $X$ and $Y$)} 
implicitly sets $\mathcal{S}$'s randomness $a:=\alpha$, picks $b \sample \mathbb{Z}_p$. 
\begin{itemize}
    \item RO queries: sets $h := g^\beta$; 
        for $z_i \notin Y$, picks $r_i \sample \mathbb{Z}_p$ and sets $\mathsf{H}(z_i):= h_i = g^{r_i}$; 
        for $z_i \in Y$, 
        picks $d_i, e_i \sample \mathbb{Z}_p$, sets $\mathsf{H}(z_i):= h_i = h^{d_i} \cdot g^{e_i}$.  

    \item $\mathcal{A}$ computes $f_j = (g^\alpha)^{r_i}$ for $x_j \in X - X \cap Y$, 
        computes $f_k = g^{\gamma d_k + e_k}$ for $k \in [m]$, outputs $(f_1, \dots, f_n)$; 
        then prepares $f_\ell = g^{\gamma d_\ell + e_\ell}$ for $\ell \in [n-m]$, 
        outputs the permutation of $(\{f_k^b\}_{k \in [m]}, \{f_\ell^b\}_{\ell \in [n-m]})$.          
\end{itemize}

If $(g, g^\alpha, g^\beta, g^\gamma)$ is a DDH tuple, then the simulation is identical to hybrid 1, 
else it is identical to hybrid 2. 
\end{trivlist}
\end{proof}





\section{Commutative Function}\label{sec:cf}
A commutative function (CF) scheme is a tuple of five algorithms $(\mathsf{Setup},\mathsf{F}^1,\mathsf{F}^2,\mathsf{S},\mathsf{E})$:
\begin{itemize}
\item $\mathsf{Setup}(1^\kappa)$: on input the security parameter $\kappa$ outputs public parameters $pp$. 

\item $\mathsf{F}^1(pp,s)$: on input public parameter $pp$ and a secret $s\in S$, outputs a half-key $p\in P$.

\item $\mathsf{F}^2(p, s)$: on input a half-key $p$ and a secret $s \in S$, outputs a noisy-key $w \in P$.

\item $\mathsf{S}(w)$: on input a noisy-key $w$, outputs a signal $\sigma \in \Sigma$. 

\item $\mathsf{E}(w,\sigma)$: on input a noisy-key $w$ and a signal $\sigma$, outputs a key $k \in K$. 
\end{itemize}
where $(\mathsf{Setup},\mathsf{F}^1,\mathsf{F}^2)$ are probabilistic algorithm, and $(\mathsf{S},\mathsf{E})$ are deterministic algorithm.

\begin{trivlist}
\item \textbf{Correctness.} For any $pp \leftarrow \mathsf{Setup}(1^\kappa)$, 
    any $s_1,s_2\leftarrow S$, $p_1 \leftarrow \mathsf{F}^1(pp,s_1)$, $p_2 \leftarrow \mathsf{F}^1(pp,s_2)$, any $w_1 \leftarrow \mathsf{F}^2(p_2,s_1)$, $w_2 \leftarrow \mathsf{F}^2(p_1,s_2)$, any $i\in \{0,1\}, \sigma_i \leftarrow \mathsf{S}(w_i)$, 
    it holds that $k_1=\mathsf{E}(w_1,\sigma_i ) = \mathsf{E}(w_2,\sigma_i)=k_2$ with overwhilming probability. 
\end{trivlist}

\subsection{Key Exchange from Commutative Functions}
As shown in Figure \ref{fig:ke-from-commutative-functions}.
\begin{figure}[!hbtp]
\begin{framed}
\begin{minipage}[center]{\textwidth}
\begin{trivlist}
\item \textbf{Parameters:} 
\begin{itemize}
    \item Common input: A CF scheme $(\mathsf{Setup},\mathsf{F}^1,\mathsf{F}^2,\mathsf{S},\mathsf{E})$, public parameter $pp\leftarrow \mathsf{Setup}(1^\kappa)$. 
\end{itemize}

\item \textbf{Protocol:}

\begin{enumerate}
\item $\mathcal{R}$ picks $s_2 \sample S$, and computes $p_2:\leftarrow \mathsf{F}^1(pp,s_2)$, 
    then sends $p_2$ to $\mathcal{S}$. 

\item $\mathcal{S}$ picks $s_1 \sample S$, and computes $p_1:\leftarrow \mathsf{F}^1(pp,s_1)$, 
    then sends $p_1$ to $\mathcal{S}$. 

\item $\mathcal{R}$ computes $w_2\leftarrow \mathsf{F}^2(p_1,s_2)$ and $\sigma:= \mathsf{S}(w_2)$, 
    then sends $\sigma$ to $\mathcal{S}$. 

\item $\mathcal{S}$ computes $w_1\leftarrow \mathsf{F}^2(p_2,s_1)$. 

\item $\mathcal{R}$ computes $k:=\mathsf{E}(w_2,\sigma)$ and $\mathcal{S}$ computes $k:=\mathcal{E}(w_1,\sigma)$.  
\end{enumerate}
\end{trivlist}
\end{minipage}
\end{framed}
\caption{Key Exchange from Commutative Functions}\label{fig:ke-from-commutative-functions}
\end{figure} 
	
\subsection{Instantiation of CF}

\subsubsection{DDH}
\begin{itemize}
\item $\mathsf{Setup}(1^\kappa)$: on input $1^\kappa$, output a description of a cyclic group $\mathbb{G}$, its order $q$, and a generator $g$, let $pp=(\mathbb{G},q,g)$.

\item $\mathsf{F}^1(pp,s)$: on input public parameter $pp=(\mathbb{G},q,g)$ and a secret $s\in \mathbb{Z}_q$, outputs a half-key $p:= g^s \in \mathbb{G}$.

\item $\mathsf{F}^2(p, s)$: on input a half-key $p$ and a secret $s \in \mathbb{Z}_q$, outputs a noisy-key $p^s \in \mathbb{G}$.

\item $\mathsf{S}(w)$: on input a noisy-key $w\in \mathbb{G}$, outputs a signal $\sigma :=\perp$. 

\item $\mathsf{E}(w,\sigma)$: on input a noisy-key $w\in \mathbb{G}$ and a signal $\sigma=\perp$, outputs a key $k:=w \in \mathbb{G}$. 
\end{itemize}


\subsubsection{DXL12}
\begin{itemize}
\item $\mathsf{Setup}(1^\kappa)$: on input $1^\kappa$, output $pp=(q,n,\alpha, M)$ where $(q,n,\alpha)$ are public parameter in lattice and $M\in \mathbb{Z}_q^{n\times n}$ is a random matrix. 

\item $\mathsf{F}^1(pp,s)$: on input public parameter $pp=(q,n,\alpha,M)$ and a secret $s\leftarrow \mathcal{D}_{\mathbb{Z}^n,\alpha q} $, selects a noise $e\leftarrow \mathcal{D}_{\mathbb{Z}^n,\alpha q}$, outputs a half-key $p:= Ms +2e \in \mathbb{Z}_q^n$.

\item $\mathsf{F}^2(p, s)$: on input a half-key $p$ and a secret $s\leftarrow \mathcal{D}_{\mathbb{Z}^n,\alpha q} $, selects a noise $e\leftarrow \mathcal{D}_{\mathbb{Z},\alpha q}$, outputs a noisy-key $w:= ps+2e \in \mathbb{Z}_q$.

\item $\mathsf{S}(w)$: on input a noisy-key $w$, run $S$ algorithm of robust extractors in DXL12 and outputs $\sigma \in \{0,1\}$. 

\item $\mathsf{E}(w,\sigma)$: on input a noisy-key $w$ and a signal $\sigma$, run $E$ algorithm of robust extractors in DXL12 and outputs $k \in \{0,1\}$. 
\end{itemize}
	

\subsubsection{Pei14}
\begin{itemize}
\item $\mathsf{Setup}(1^\kappa)$: on input $1^\kappa$, output $pp=(a, g)$, where $a\in R_q$ and $g\in R$ as description in Pei14.

\item $\mathsf{F}^1(pp,s)$: on input public parameter $pp=(a,g)$ and a secret $s\leftarrow \chi $, selects a noise $e\leftarrow \chi$, outputs a half-key $p:= as +e \in R_q$.

\item $\mathsf{F}^2(p, s)$: on input a half-key $p\in R_q$ and a secret $s\in R$, selects a noise $e\leftarrow \chi$, outputs a noisy-key $w:= gps+e \in R_q$.

\item $\mathsf{S}(w)$: on input a noisy-key $w$, let $\sigma:=\langle \bar{v}\rangle_2\in R_2$, where $\bar{v}=dbl(v)$ as defined in Pei14.

\item $\mathsf{E}(w,\sigma)$: on input a noisy-key $w$ and a signal $\sigma$, run $rec$ algorithm of Pei14 and outputs $k:= rec(w,\sigma) \in R_2$. 
\end{itemize}	



\subsection{PSU from CF}\label{sec:psu-from-cf}
We construct a linear complexity PSU protocol from the CF in Figure~\ref{fig:CF-PSU}.  
\begin{figure}[!hbtp]
\begin{framed}
\begin{minipage}[center]{\textwidth}
\begin{trivlist}
\item \textbf{Parameters:} 
\begin{itemize}
    \item Common input: A CF scheme $(\mathsf{Setup},\mathsf{F}^1,\mathsf{F}^2,\mathsf{S},\mathsf{E})$, public parameter $pp\leftarrow \mathsf{Setup}(1^\kappa)$.

    \item Input of sender $\mathcal{S}$: $X = \{x_1, \dots, x_n\}$.

    \item Input of receiver $\mathcal{R}$: $Y = \{y_1, \dots, y_n\}$. 
\end{itemize}

\item \textbf{Protocol:}

\begin{enumerate}
\item $\mathcal{R}$ picks $b \sample S$, then sends $(\mathsf{F}^1(\mathsf{H}(y_1),b), \dots, \mathsf{F}^1(\mathsf{H}(y_n),b))$ to $\mathcal{S}$. 

\item $\mathcal{S}$ picks $a \sample S$, 
    sends $(\mathsf{F}^1(\mathsf{H}(x_1),a), \dots, \mathsf{F}^1(\mathsf{H}(x_n),a))$ to $\mathcal{R}$; 
    he also computes $(w_1,\dots,w_n):= (\mathsf{F}^2(\mathsf{F}^1(\mathsf{H}(y_1),b),a), \dots, \mathsf{F}^2(\mathsf{F}^1(\mathsf{H}(y_n),b),a))$, $\sigma_i:=\mathsf{S}(w_i)$ and $\Gamma:= \{k_i| k_i:=\mathsf{E}(w_i,\sigma_i),i\in [n]\}$. Then, he sends its permutation $\Gamma$ to $\mathcal{R}$. 


\item $\mathcal{R}$ computes $(w_1',\dots,w_n'):= (\mathsf{F}^2(\mathsf{F}^1(\mathsf{H}(x_1),a),b), \dots,\mathsf{F}^2(\mathsf{F}^1(\mathsf{H}(x_n),a),b))$, $\sigma_i':=\mathsf{S}(w_i')$, and computes $k_i':=\mathsf{E}(w_i',\sigma_i'), i\in [n]$. 
    then sets $v_i = 0$ iff $k_i'$ is not in $\Gamma$. 

\item $\mathcal{R}$ with select vector $(v_1, \dots, v_n)$ and $\mathcal{S}$ with input $\{(x_i, \bot)\}_{i \in [n]}$ 
    engage in one-sided OT. 

\item $\mathcal{R}$ obtains $X - X \cap Y$, and thus obtains the union $X \cup Y$.  
\end{enumerate}
\end{trivlist}
\end{minipage}
\end{framed}
\caption{CF-PSU}\label{fig:CF-PSU}
\end{figure}










