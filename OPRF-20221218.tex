% switch between git and local
\newif\ifGit
\Gittrue
%\Gitfalse

% set path
\ifGit
    \newcommand{\stylepath}{.}
    \newcommand{\bibpath}{.}
\else
    \newcommand{\stylepath}{/Users/yuchen/Documents/Research/PaperEngine/paper}
    \newcommand{\bibpath}{/Users/yuchen/Documents/Research/PaperEngine/bib}
\fi

% switch between Chinese and English
\newif\ifChinese
%\Chinesetrue
\Chinesefalse


\documentclass[a4paper,10pt]{article}


\usepackage{fullpage}
\usepackage{caption}
\usepackage{latexsym}
\usepackage{amsmath, amsthm, amsfonts}
\usepackage{mathrsfs}
\usepackage{graphicx}
\usepackage{subfigure}
\usepackage{xcolor}
\usepackage{colortbl}
\usepackage{soul}
\usepackage{enumerate}
\usepackage{enumitem}
\usepackage{setspace}
\usepackage{framed}
\usepackage{boxedminipage}
\usepackage{cancel,siunitx}
\usepackage{amssymb}% http://ctan.org/pkg/amssymb
\usepackage{pifont}% http://ctan.org/pkg/pifont
\usepackage{makecell}
\usepackage{tabularx}
\usepackage{multirow}
\usepackage{threeparttable}
\usepackage[ruled,vlined,linesnumbered]{algorithm2e}
\usepackage{url}
\usepackage{float}
\usepackage[numbers,sectionbib]{natbib}
\usepackage{mdframed}
\usepackage[normalem]{ulem} % 批注
\usepackage{autobreak}

\makeatletter
    \renewcommand\@biblabel[1]{{[#1]\hfill}}
\makeatother
\setlength{\bibsep}{0.5ex}
\usepackage[colorlinks, anchorcolor=blue, linkcolor=blue, citecolor=blue, urlcolor=blue]{hyperref}
\usepackage[title,titletoc,header]{appendix}
% \usepackage{authblk}
% \renewcommand*{\Affilfont}{\small}

% \usepackage{tikz}     % 绘图包

\usepackage{tikz, pgfplots}
\usetikzlibrary{matrix, fit, calc, matrix, patterns, backgrounds, arrows, 
shapes, chains, fit, decorations, intersections, positioning}

\tikzset{
    shapenode/.style = {draw, rectangle, fill=none, minimum size=0.5cm, minimum height=0.6cm, minimum width=0.6cm, 
    auto, node distance=0em, font=\normalsize}, 
    roundnode/.style = {draw, rectangle, rounded corners=0.5em, inner sep = 0em, 
      node distance=0em, minimum height=1.5em},
    rectanglenode/.style = {draw, rectangle, inner sep = 0em, node distance=0em},
    ellipsenode/.style = {draw, ellipse, inner sep = 0em, node distance=0em},
    textnode/.style  = {draw=none, fill=none, rectangle, node distance=0em, font=\normalsize},
    smalltextnode/.style  = {draw=none, fill=none, rectangle, minimum size=0.5cm, auto, node distance=0em, font=\small},
    circlenode/.style = {draw, circle, auto, inner sep = 0em, node distance=0em, minimum size=1.5em}, 
    dotnode/.style = {draw, fill=black, circle, auto, minimum size=0.1em, inner sep = 0em, node distance=0em},  
    connect/.style = {black,->}, 
    every text node part/.style={align=center}
}

% define the colors
\definecolor{lightmidnightblue}{rgb}{0.2, 0.2, 0.7}
\definecolor{darkgreen}{rgb}{0.0,0.5,0.0}

\newcommand{\blue}[1]{{\color{blue} #1}}
\newcommand{\cyan}[1]{{\color{cyan} #1}}
\newcommand{\purple}[1]{{\color{purple} #1}}
\newcommand{\yellow}[1]{{\color{yellow} #1}}
\newcommand{\darkred}[1]{{\color{darkred} #1}}
\newcommand{\gray}[1]{{\color{gray} #1}}
\newcommand{\red}[1]{{\color{red} #1}}
\newcommand{\green}[1]{{\color{green}#1}}
\newcommand{\magenta}[1]{{\color{magenta} #1}}
\newcommand{\black}[1]{{\color{black} #1}}
\newcommand{\orange}[1]{{\color{orange} #1}}
\newcommand{\mdblue}[1]{{\color{lightmidnightblue} #1}}

\colorlet{lightgray}{gray!20}
\colorlet{darkgray}{gray!100}

% define notions
\renewcommand{\P}{\mathbf{P}}
\newcommand{\NP}{\mathbf{NP}}
\newcommand{\coNP}{\textup{co-}\mathbf{NP}}

% define shorthands
\newcommand{\define}{\stackrel{\textup{def}}{=}}

\newcommand{\iO}{i\mathcal{O}}
\newcommand{\diO}{di\mathcal{O}}
\newcommand{\sample}{\xleftarrow{\textup{\tiny R}}}
\newcommand{\expect}{\mathbb{E}}
\newcommand{\entropy}{\mathsf{H}}
\newcommand{\minentropy}{\mathsf{H}_\infty}
\newcommand{\avminentropy}{\tilde{\mathsf{H}}_\infty}
\newcommand{\AdvA}{\mathsf{Adv}_\mathcal{A}(\lambda)}

\newcommand{\outV}{\text{out}_V}
\newcommand{\viewV}{\text{view}_V}
\newcommand{\viewVstar}{\text{view}_{V^*}}

\newcommand{\Oderive}{\mathcal{O}_\mathsf{derive}^\Phi}
\newcommand{\Oinv}{\mathcal{O}_\mathsf{inv}}
\newcommand{\Oleak}{\mathcal{O}_\mathsf{leak}(\cdot)}
\newcommand{\Ovefy}{\mathcal{O}_\mathsf{vefy}(\cdot,\cdot)}
\newcommand{\Otag}{\mathcal{O}_\mathsf{tag}(\cdot)}
\newcommand{\Osign}{\mathcal{O}_\mathsf{sign}}
\newcommand{\Odec}{\mathcal{O}_\mathsf{dec}}
\newcommand{\Odecaps}{\mathcal{O}_\mathsf{decaps}}
\newcommand{\OregH}{\mathcal{O}_\mathsf{regH}}
\newcommand{\OregC}{\mathcal{O}_\mathsf{regC}}
\newcommand{\Oreveal}{\mathcal{O}_\mathsf{reveal}}
\newcommand{\Oibedecaps}{\mathcal{O}_\mathsf{decaps}(\cdot, \cdot)}
\newcommand{\Osim}{\mathcal{O}_\mathsf{sim}}
\newcommand{\Oenc}{\mathcal{O}_\mathsf{enc}}
\newcommand{\Oext}{\mathcal{O}_\mathsf{ext}}
\newcommand{\Oconstrain}{\mathcal{O}_\mathsf{constrain}}
\newcommand{\Oevalempty}{\mathcal{O}_\mathsf{eval}(\$)}

\newcommand{\Oracle}{\mathcal{O}}
\newcommand{\TBA}{\blue{To be add more ...}}

\newcommand{\poly}{\textup{poly}}
\newcommand{\polylog}{\textup{polylog}}
\newcommand{\loglog}{\textup{loglog}}
% \newcommand{\MA}{\mathbf{MA}}
% \newcommand{\PSAPCE}{\mathbf{PSPACE}}
% \newcommand{\NEXT}{\mathbf{NEXT}}

% define theorem enviroments
\ifChinese

\newtheorem{theorem}{定理}[section]
\newtheorem{lemma}[theorem]{引理}
\newtheorem{claim}[theorem]{断言}
\newtheorem{proposition}[theorem]{命题}
\newtheorem{corollary}[theorem]{推论}
\newtheorem{definition}{定义}[section]
\newtheorem{example}{例}[section]
\newtheorem{remark}{注}[section]
\newenvironment{solution}{\begin{proof}[\textbf 解]}{\end{proof}}
\renewcommand{\proofname}{\textbf 证明}

\else
% define theorem enviroments
\theoremstyle{plain}% default
\newtheorem{theorem}{Theorem}[section]
\newtheorem{lemma}[theorem]{Lemma}
\newtheorem{claim}[theorem]{Claim}
\newtheorem{proposition}[theorem]{Proposition}
\newtheorem{corollary}[theorem]{Corollary}

\theoremstyle{definition}
\newtheorem{definition}{Definition}[section]
\newtheorem{conjecture}{Conjecture}[section]
\newtheorem{example}{Example}[section]
\newtheorem{note}{Note}[section]
\newtheorem{assumption}{Assumption}[section]
\newtheorem{construction}{Construction}[section]
\newtheorem{exercise}{Exercise}[section]
\newtheorem{discussion}{Discussion}[section]

\theoremstyle{remark}
\newtheorem{remark}{Remark}[section]
\newtheorem{case}{Case}[section]
\newtheorem{observation}{Observation}[section]
\fi


% define references
\renewcommand\bibname{References}
\renewcommand{\bibfont}{\small}

\newcommand{\cmark}{\ding{51}}%
\newcommand{\xmark}{\ding{55}}%

\newcommand{\redul}{\bgroup\markoverwith{\textcolor{red}{\rule[-0.5ex]{2pt}{0.4pt}}}\ULon} % 红色下划线
\newcommand{\blueul}{\bgroup\markoverwith{\textcolor{blue}{\rule[-0.5ex]{2pt}{0.4pt}}}\ULon} % 蓝色下划线
\newcommand{\blackul}{\bgroup\markoverwith{\textcolor{black}{\rule[-0.5ex]{2pt}{0.4pt}}}\ULon} % 黑色下划线

% define new commands
\newcommand{\myindent}{\hspace{1em}}
\newcommand{\keywords}[1]{\par\addvspace\baselineskip\noindent\keywordname\enspace\ignorespaces#1}

\newcommand{\hatf}{\hat{f}}

\newenvironment{mytrivlist}{\begin{trivlist}\itemsep 1pt \parskip 0pt \parsep 0pt}{\end{trivlist}}
\newenvironment{myitemize}{\begin{itemize}\itemsep 1pt \parskip 0pt \parsep 0pt}{\end{itemize}}
\newenvironment{myenumerate}{\begin{enumerate}\itemsep 1pt \parskip 0pt \parsep 0pt}{\end{enumerate}}

\newenvironment{noteblock}[0]{%
\setbeamercolor{block body}{bg=gray!20,fg=black}
\begin{block}{}}
{\end{block}}

\renewcommand\bibname{References}
\renewcommand{\bibfont}{\small}




\newcommand{\FuncOT}{\mathcal{F}_\mathsf{OT}}
\newcommand{\FuncMQRPMT}{\mathcal{F}_\mathsf{mqRPMT}}
\newcommand{\FuncOPRF}{\mathcal{F}_\mathsf{OPRF}}
\newcommand{\FuncPSU}{\mathcal{F}_\mathsf{PSU}}
\newcommand{\FuncPSI}{\mathcal{F}_\mathsf{PSI}}
\newcommand{\FuncPSO}{\mathcal{F}_\mathsf{PSO}}

\newcommand{\SimR}{\mathsf{Sim}_\mathcal{R}}
\newcommand{\SimS}{\mathsf{Sim}_\mathcal{S}}

% \usepackage{authblk}

\begin{document}

\thispagestyle{empty}

\title{OPRF from Input-Homomorphic Publicly Evaluable PRF}


\author{
  Yu Chen
  \thanks{Shandong University. 
        Email: \blue{\texttt{yuchen.prc@gmail.com}}}
  \and
  % Minglang Dong
  % \thanks{Shandong University. 
  %       Email: \blue{\texttt{minglang\_dong@mail.sdu.edu.cn}}}
  \and
  Yamin Liu
  \thanks{Shield Lab, Huawei. Email: \blue{\texttt{liuyamin3@huawei.com}}}
}



\date{}

\maketitle

\begin{abstract}

Oblivious pseudorandom function (OPRF), a counterpart of PRF in distributed setting, 
has become a ubiquitous primitive used in secure multi-party computation.  
In this work, we provide a generic construction of OPRF from publicly evaluable PRF with input-homomorphic property. 
By plugging PEPRF from efficient group action (EGA) to the generic construction, we obtain a OPRF with plausible post-quantum security.  
           
\begin{trivlist}
\item \textbf{Keywords:} OPRF, homomorphic, PEPRF
\end{trivlist}
\end{abstract}

\thispagestyle{empty}

\newpage
\setcounter{tocdepth}{2}
\tableofcontents
\newpage

\pagestyle{plain}
\pagenumbering{arabic}
\setcounter{page}{1}
    
    
\section{Introduction}\label{sec:introduction}
Pseudorandom functions (PRFs) is a core cryptographic primitive with numerous applications. 
Two decades later, Freedman et al.~\cite{FIPR-TCC-2005} introduce its counterpart in the distributed setting, 
namely oblivious pseudorandom functions (OPRF). 
In a nutshell, OPRF is an interactive 2-party protocol between a client with input $\{x_i\}_{i \in [n]}$,  
and a server with secret key $k$. 
It allows the client learn $\{F_k(x_i)\}_{i \in [n]}$ without anything else, and the server learns nothing. 
Formally, from the perspective of 2-party secure computation, an OPRF is a protocol securely 
fulfills the functionality $\mathcal{F}_\text{oprf} = (\bot, \{F_k(x_i)\}_{i \in [n]})$.   
OPRFs have shown to be a central primitive for building oblivious keyword search~\cite{FIPR-TCC-2005}, 
private set intersection~\cite{KKRT-CCS-2016, HL-JoC-2010, CM-CRYPTO-2020}, 
secure data de-duplication~\cite{CCGS-FC-2019}, 
password-authenticated key exchange (PAKE)~\cite{JKK-ASIACRYPT-2014,JKX-EUROCRYPT-2018}, 
private information retrieval~\cite{FIPR-TCC-2005}, secure pattern matching~\cite{FHV-ICALP-2013}, 
cloud key management~\cite{JKR-CCS-2019}. 

% Quantum security of PRF (see BCR)
So far, except the OPRFs based on symmetric primitives, all known efficient OPRF constructions 
rely on number-theoretical assumptions, which are not post-quantum secure. 
Recently, Boneh et al.~\cite{BKW-ASIACRYPT-2020} propose an isogency-based OPRF. 
Albrecht et al.~\cite{ADDS-PKC-2021} present a lattice-based OPRF with \emph{verifiable} property. 
These two constructions are post-quantum secure, but both them only serves as nice feasibility results, 
rather than practical results with good performance.  


\subsection{Motivation}
Motivated by the state of affairs, we are intrigued to know:
\begin{center}
\emph{Is there a generic construction of OPRF which admits for efficient instantiation?} 
\end{center}  

\subsection{Our Contribution}
In this work, we make positive answers to the aforementioned questions. 
We present a generic construction of OPRF from PEPRF with input-homomorphic property, 
then give an efficient post-quantum secure instantiation from EGA.     

%We summarize our contribution as below. 

% \begin{trivlist}
% \item \textbf{Generic construction from PEPRF.} We first present a generic construction of OPRF 
%     from PEPRF with input-homomorphic property.   

% \item \textbf{Efficient instantiations from lattice.}  
% \end{trivlist}


% \subsection{Technical Overview}



% \subsection{Related Works}


\section{Preliminaries}\label{sec:preliminaries}

\subsection{Notation}
We use $\kappa$ and $\lambda$ to denote the computational and statistical parameter respectively. 
Let $\mathbb{Z}_n$ be the set $\{0,1, \dots, n-1\}$, $\mathbb{Z}_n^* = \{x \in \mathbb{Z}_n \mid \text{gcd}(x, n) = 1\}$. 
We use $[n]$ to denote the set $\{1, \dots, n\}$, 
and use $\mathsf{Perm}[n]$ to denote all the permutation over the set $\{1, \dots, n\}$. 
We assume that every set $X$ has a default order (e.g. lexicographical order), 
and write it as $X = \{x_1, \dots, x_n\}$.
For a set $X$, we use $|X|$ to denote its size and use $x \sample X$ to denote sampling $x$ uniformly at random from $X$.
We use $(x_1, \dots, x_n)$ to denote a vector, whose $i$th element is $x_i$.  
A function is negligible in $\kappa$, written $\mathsf{negl}(\kappa)$, 
if it vanishes faster than the inverse of any polynomial in $\kappa$.
A probabilistic polynomial time (PPT) algorithm is a randomized algorithm that runs in polynomial time. 

\subsection{Weak Pseudorandom EGA}\label{subsec:EGA}
We begin by recalling the definition of a group action. 
\begin{definition}[Group Actions]
A group $\mathbb{G}$ is said to \emph{act on} a set $X$ if there is a map $\star: \mathbb{G} \times X \rightarrow X$ 
that satisfies the following two properties: 
\begin{enumerate}
    \item Identity: if $e$ is the identity element of $\mathbb{G}$, then for any $x \in X$, we have $e \star x = x$. 
    \item Compatibility: for any $g, h \in \mathbb{G}$ and any $x \in X$, 
        we have $(gh) \star x = g \star (h \star x)$. 
\end{enumerate}
\end{definition}

From now on, we use the abbreviated notation $(\mathbb{G}, X, \star)$ to denote a group action. 
If $(\mathbb{G}, X, \star)$ is a group action, for any $g \in \mathbb{G}$ the map $\phi_g: x \mapsto g \star x$ 
defines a permutation of $X$. 

We then define an effective group action (EGA)~\cite{AFMP-ASIACRYPT-2020} as follows. 
\begin{definition}[Effective Group Actions]
A group action $(\mathbb{G}, X, \star)$ is \emph{effective} if the following properties are satisfied:  
\begin{enumerate}
\item The group $\mathbb{G}$ is finite and there exist PPT algorithms for: 
\begin{enumerate}
    \item Membership testing, i.e., to decide if a given bit string represents a valid group element in $\mathbb{G}$. 
    \item Equality testing, i.e., to decide if two bit strings represent the same group element in $\mathbb{G}$.
    \item Sampling, i.e., to sample an element $g$ from a uniform (or statistically close to) distribution on $\mathbb{G}$.  
    \item Operation, i.e., to compute $gh$ for any $g, h \in \mathbb{G}$. 
    \item Inversion, i.e., to compute $g^{-1}$ for any $g \in \mathbb{G}$.
\end{enumerate}

\item The set $X$ is finite and there exist PPT algorithms for: 
\begin{enumerate}
    \item Membership testing, i.e., to decide if a bit string represents a valid set element. 

    \item Unique representation, i.e., given any set element $x \in X$, 
        compute a string $\hat{x}$ that canonically represents $x$.
\end{enumerate}

\item There exists a distinguished element $x_0 \in X$, 
    called the origin, such that its bit-string representation is known.

\item There exists an efficient algorithm that given (some bit-string representations of) any $g \in \mathbb{G}$ 
    and any $x \in X$, outputs $g \star x$.
\end{enumerate} 
\end{definition}

\begin{definition}[Weak Pseudorandom EGA]
A group action $(G, X, \star)$ is weakly pseudorandom if the family of efficiently commutable permutation 
$\{\phi_g: X \rightarrow X\}_{g \in G}$ is weakly pseudorandom, i.e., 
there is no PPT adversary that can distinguish tuples of the form $(x_i, g \star x_i)$ 
from $(x_i, u_i)$ where $g \sample \mathbb{G}$ and each $x_i, u_i \sample X$.
\end{definition}


\subsection{MPC in the Semi-honest Model}
We use the standard notion of security in the presence of semi-honest adversaries. 
Let $\Pi$ be a two-party protocol for computing the function $f(x_1, x_2)$, where party $P_i$ has input $x_i$. 
We define security in the following way.
For each party $P$, let $\text{View}_P(x_1, x_2)$ denote the view of party $P$ 
during an honest execution of $\Pi$ on inputs $x_1$ and $x_2$. 
The view consists of $P$'s input, random tape, and all messages exchanged as part of the $\Pi$ protocol. 

\begin{definition}
2-party protocol $\Pi$ securely realizes $f$ in the presence of semi-honest adversaries if 
there exists a simulator $\mathsf{Sim}$ such that for all inputs $x_1, x_2$ and all $i \in \{1,2\}$:
\begin{gather*}
\mathsf{Sim}(i, x_i, f(x_1, x_2)) \approx_c \text{View}_{P_i}(x_1, x_2)
\end{gather*}
\end{definition}
Roughly speaking, a protocol is secure if the party with $x_i$ 
learns no more information other than $f(x_1, x_2)$ and $x_i$.  








\section{Review of Pseudorandom Function}
In this section, we first recall the notion of standard pseudorandom functions (PRFs)~\cite{GGM-JACM-1986}, 
then recap its counterpart in the public-key setting, 
namely publicly evaluable pseudorandom functions~\cite{CZ-SCN-2014}. 

\subsection{Standard PRF}
\begin{definition}[PRF]
A family of PRFs consists of three polynomial-time algorithms as follows: 
\begin{itemize}
\item $\mathsf{Setup}(1^\kappa)$: on input a security parameter $\kappa$, 
    outputs public parameter $pp$ specifying $(F, K, X, Y)$, 
    where $F: K \times X \rightarrow Y$ is a family of keyed functions, 
    $K$ is the key space, $X$ is domain, and $Y$ is range.

\item $\mathsf{KeyGen}(pp)$: on input $pp$, outputs a secret key $k \sample K$. 

\item $\mathsf{Eval}(k, x)$: on input $k \in K$ and $x \in X$, outputs $y \leftarrow F(k, x)$. 
    For notation convenience, we will write $F(k, x)$ as $F_k(x)$ interchangeably. 
\end{itemize}

\begin{trivlist}
\item \textbf{Correctness:} For any $\kappa \in \mathbb{N}$, 
    any $pp \leftarrow \mathsf{Setup}(1^\kappa)$ and any $k \leftarrow \mathsf{KeyGen}(pp)$, 
    we have $F_{sk}(x) = \mathsf{Eval}(k, x)$. 

\item \textbf{Pseudorandomness.} Let $\mathcal{A}$ be an adversary against PRFs and define its advantage as:
\begin{displaymath}
    \mathsf{Adv}_\mathcal{A}(\kappa) =
        \Pr \left[ \beta' = \beta:~
        \begin{array}{l}
            pp \leftarrow \mathsf{Setup}(1^\kappa);\\
            k \leftarrow \mathsf{KeyGen}(pp);\\
            \beta \leftarrow \{0,1\}; \\
            \beta' \leftarrow \mathcal{A}^{\mathcal{O}_\mathsf{ror}(\beta, \cdot)}(\kappa);
        \end{array} 
        \right] - \frac{1}{2},
\end{displaymath}
where $\mathcal{O}_\mathsf{ror}(\beta, \cdot)$ denotes the real-or-random oracle controlled by $\beta$, 
i.e., $\mathcal{O}_\mathsf{ror}(0, x) = F_k(x)$, $\mathcal{O}_\mathsf{ror}(1, x) = \mathsf{H}(x)$ 
(here $\mathsf{H}$ is chosen uniformly at random from all the functions 
from $D$ to $R$\footnote{To efficiently simulate access to a uniformly random function $\mathsf{H}$ from $D$ to $R$, 
one may think of a process in which the adversary's queries to $\mathcal{O}_\mathsf{ror}(1, \cdot)$ 
are ``lazily'' answered with independently and randomly chosen elements in $R$, 
while keeping track of the answers so that queries made repeatedly are answered consistently.}). 
$\mathcal{A}$ can adaptively access the oracle $\mathcal{O}_\mathsf{ror}(\beta, \cdot)$ polynomial many times.
We say that $F$ is pseudorandom if for any PPT adversary $\mathsf{Adv}_\mathcal{A}(\kappa)$ is negligible in $\kappa$. 
We refer to such security as full PRF security. 
\end{trivlist}
\end{definition}

Sometimes the full PRF security is not needed and it is sufficient 
if the function cannot be distinguished from a uniform random one when challenged on random inputs. 
The formalization of such relaxed requirement is \emph{weak pseudorandomness}, 
which is defined the same way as pseudorandomness except that the inputs of 
oracle $\mathcal{O}_\mathsf{ror}(\beta, \cdot)$ are uniformly chosen from $X$ by the challenger
instead of adversarially chosen by $\mathcal{A}$. 
PRFs that satisfy weak pseudorandomness are referred to as \emph{weak PRFs}.  

\begin{remark}
In a standard PRF, it is always harmless to explicitly introduce a public key, 
which includes the information related to secret key that can be made public. 
For example, in the Naor-Reingold PRF~\cite{NR-JACM-2004} based on the DDH assumption: 
$\mathsf{F}_{\vec{a}}(x) = (g^{a_0})^{\prod_{x_i=1}a_i}$, 
where $\vec{a} = (a_0, a_1, \dots, a_n) \in \mathbb{Z}_p^n$ is the secret key, 
$g^{a_i}$ for $1 \leq i \leq n$ can be safely published as the public key.  
If no information can be made public, one can simply set $pk = \bot$. 
Looking ahead, such treatment allows us get a unified syntax of PRF 
in both $\mathsf{minicrypt}$ and $\mathsf{cryptmania}$.     
\end{remark}

\subsection{Publicly Evaluable PRF}
\begin{definition}[Publicly Evaluable PRFs]\label{def:PEPRF}
A family of PEPRFs consists of four polynomial-time algorithms as follows:
\begin{itemize} 
    \item $\mathsf{Setup}(1^\kappa)$: on input a security parameter $\kappa$, 
        output public parameter $pp$ specifying $(F, PK, SK, X, \allowbreak Y, L, W)$, 
        where $F: SK \times X \rightarrow Y \cup \bot$ is family of keyed functions, 
        $SK$ is the secret key space, $PK$ is the corresponding public key space, 
        $X$ is domain, $Y$ is range, 
        $L \subseteq X$ is a NP language defined by some hard relation $\mathsf{R}_L$, 
        and $W$ is the space of associated witnesses. 
        Here, we assume that $\mathsf{R}_L$ is efficiently sampleable, 
        that is, there exists a PPT algorithm $\mathsf{SampRel}$ 
        which on input random coins $r$, output a random tuple $(x, w) \in \mathsf{R}_L$.   

    \item $\mathsf{KeyGen}(pp)$: on input $pp$, output a secret key $sk$ and 
        an associated public key $pk$.\footnote{In a standard PRF, 
        it is also harmless to explicitly introduce a public key, 
        which includes the information related to secret key that can be made public. 
        For example, in the Naor-Reingold PRF~\cite{NR-JACM-2004} based on the DDH assumption: 
        $\mathsf{F}_{\vec{a}}(x) = (g^{a_0})^{\prod_{x_i=1}a_i}$, where
        $\vec{a} = (a_0, a_1, \dots, a_n) \in \mathbb{Z}_p^n$ is the secret key, 
        $g^{a_i}$ for $1 \leq i \leq n$ can be safely published as the public key.  
        If no information can be made public, one can always assume $pk = \{\bot\}$.}     
    
    \item $\mathsf{PrivEval}(sk, x)$: on input $sk$ and $x \in X$, 
        output $y \in Y \cup \bot$.  

    \item $\mathsf{PubEval}(pk, x, w)$: on input $pk$ and $x \in L$ together with a witness $w \in W$, 
        output $y \in Y$.  
\end{itemize}
\end{definition}

\begin{trivlist}
\item \textbf{Correctness:} For any $\lambda \in \mathbb{N}$, 
    any $pp \leftarrow \mathsf{Setup}(1^\lambda)$ and any $(pk, sk) \leftarrow \mathsf{KeyGen}(pp)$, we have:
    \begin{eqnarray*}
        & \forall x \in X: & F_{sk}(x) = \mathsf{PrivEval}(sk, x)\\
        & \forall x \in L \text{~with a witness~} w: & F_{sk}(x) = \mathsf{PubEval}(pk, x, w)
    \end{eqnarray*} 

\item \textbf{Input homomorphic:} Suppose $L \subseteq X$ and $Y$ are finite abelian groups. 
    $F$ is input homomorphic if for any $x_1, x_2 \in L$ we have $F_{sk}(x_1+x_2) = F_{sk}(x_1)+F_{sk}(x_2)$.    

\item \textbf{Weak pseudorandomness:} Let $\mathcal{A}$ be an adversary 
    against PEPRFs and define its advantage as:
    \begin{displaymath}
        \mathsf{Adv}_\mathcal{A}(\lambda) =
            \Pr \left[\beta = \beta':~ 
                \begin{array}{ll}
                    & pp \leftarrow \mathsf{Setup}(1^\lambda);\\
                    & (pk, sk) \leftarrow \mathsf{KeyGen}(pp);\\
                    & \{r_i^* \sample R, (x_i^*, w_i^*) \leftarrow \mathsf{SampRel}(r_i^*)\}_{i=1}^{p(\lambda)};\\
                    & \beta \leftarrow \{0,1\};\\   
                    & \beta' \leftarrow \mathcal{A}(pp, pk,  
                    \{x_i^*, \mathcal{O}_\mathsf{ror}(\beta, x_i^*)\}_{i=1}^{p(\lambda)});
                \end{array} 
            \right] - \frac{1}{2},
    \end{displaymath}
\end{trivlist}
where $p(\kappa)$ is any polynomial, 
$\mathcal{O}_\mathsf{ror}(\beta, \cdot)$ is the real-or-random oracle. 
We say that $F$ is weak pseudorandom if for any PPT adversary 
$\mathcal{A}$ its advantage function $\mathsf{Adv}_\mathcal{A}(\kappa)$ is negligible in $\kappa$. 

Looking ahead, to simplify the security proof we will use a one-time flavor definition 
that $\mathcal{A}$ only queries $\mathcal{O}_\mathsf{ror}$ once when arguing the security of PEPRF constructions. 
A standard hybrid argument shows that such seemingly restricted definition is actually equivalent to 
the standard multi-query definition.

\begin{remark}
    Different from standard PRFs, PEPRFs require the existence 
    of a language $L \subseteq X$ as well as a public evaluation algorithm.
    Due to this strengthening on functionality, 
    we cannot hope to achieve standard pseudorandomness, 
    and hence weak pseudorandomness is the best we can achieve.   
\end{remark} 


\section{OPRF from Input-Homomorphic PEPRF}\label{sec:OPRF-from-PEPRF}
\subsection{Definition of OPRF}
In Figure~\ref{figure:func-oprf}, 
we adapt the definion of multi-point OPRF~\cite{CM-CRYPTO-2020} under unified PRF syntax.   

\begin{figure}[!hbtp]
\begin{framed}
\begin{minipage}[center]{\textwidth}
\begin{trivlist}
\item \textbf{Parameters:} A PRF $F: SK \times X \rightarrow Y$, and two parties: server $P_1$ and client $P_2$

\item \textbf{Functionality:}
\begin{enumerate}
\item Generate a key pair $(pk, sk)$ for PRF. Give $(pk, sk)$ to $P_1$ and give $pk$ to $P_2$. 
\item Wait for input $(x_1, \dots, x_n)$ from $P_2$. Give $(F_{sk}(x_1), \dots, F_{sk}(x_n))$ to $P_2$
\end{enumerate}
\end{trivlist}
\end{minipage}
\end{framed}
\caption{Ideal functionality $\FuncMQRPMT$ for multi-point OPRF}\label{figure:func-oprf}
\end{figure}

\subsection{Construction from Input-Homomorphic PEPRF}
In Figure~\ref{figure:OPRF-from-PEPRF}, 
we show how to build OPRF $F: SK \times \{0,1\}^{\ell} \rightarrow Y$ from PEPRF $G: SK \times X \rightarrow Y$ 
and cryptographic hash function $\mathsf{H}: \{0,1\}^\ell \rightarrow L$. 

\begin{remark}
Looking ahead, $\mathsf{H}: \{0,1\}^\ell \rightarrow L$ will be modeled as a random oracle. 
This implicitly requires $L$ is dense. 
\end{remark}

\begin{figure}[!hbtp]
\begin{framed}
\begin{minipage}[center]{\textwidth}
\begin{trivlist}
\item \textbf{Parameters:} 
\begin{itemize}
    \item Common input: PEPRF $G: SK \times X \rightarrow Y$, hash function $\mathsf{H}: \{0,1\}^\ell \rightarrow L$. 

    \item Input of server $P_1$: security parameter $\kappa$.

    \item Input of client $P_2$: $X = \{x_1, \dots, x_n\} \subseteq \{0,1\}^\ell$ 
\end{itemize}

\item \textbf{Protocol:}

\begin{enumerate}
\item $P_1$ runs $pp \leftarrow \text{PEPRF}.\mathsf{Setup}(1^\kappa)$ 
    and $(pk, sk) \leftarrow \text{PEPRF}.\mathsf{KeyGen}(pp)$, 
    then sends $pp$ and $pk$ to $P_2$. 

\item $P_2$ runs $\mathsf{SampRel}(pp)$ independently $n$ times and obtains $n$ tuples $(z_i, w_i)$, 
    then sends $\{\mathsf{H}(x_i)+z_i\}_{i \in [n]}$ to $P_1$. 

\item $P_1$ computes $v_i \leftarrow \{F_{sk}(\mathsf{H}(x_i)+z_i)\}_{i \in [n]}$ and sends them to $P_2$. 
    $P_2$ computes $y_i \leftarrow v_i - \text{PEPRF}.\mathsf{PubEval}(pk, z_i, w_i)$.    
\end{enumerate}
\end{trivlist}
\end{minipage}
\end{framed}

\begin{center}
\begin{tikzpicture}
\node[textnode, name=common-input] {$G: SK \times X \rightarrow Y$, $\mathsf{H}: \{0,1\}^\ell \rightarrow X$};
\node[textnode, name=P1, below of = common-input, xshift=-15em, yshift=-1.5em] {$P_1$ (server)};
\node[textnode, below of = P1, yshift=-1em] {$\kappa$};  
\node[textnode, name=P2, below of = common-input, xshift=15em, yshift=-1.5em] {$P_2$ (client)}; 
\node[textnode, below of = P2, yshift=-1em] {$X = (x_1, \dots, x_n)$};  

\node[textnode, name = k1, below of = P1, yshift=-3em] 
    {$pp \leftarrow \text{PEPRF}.\mathsf{Setup}(1^\kappa)$\\
     $(pk, sk) \leftarrow \text{PEPRF}.\mathsf{KeyGen}(pp)$};
\draw ($(P1.east)+(5em, -3em)$) edge[->] node[above] {$pp$, $pk$} ($(P2.west)+(-5em, -3em)$);  

\node[textnode, name = k2, below of = P2, yshift=-6em] {$(z_i, w_i) \leftarrow \mathsf{SampRel}(pp)$};
\draw ($(P2.west)+(-5em, -6em)$) edge[->] node[above] {$\{\mathsf{H}(x_i)+z_i\}_{i \in [n]}$}  
                                          ($(P1.east)+(5em, -6em)$);


\draw ($(P1.east)+(5em, -9em)$) edge[->] node[above] {$\{v_i \leftarrow F_{sk}(\mathsf{H}(x_i)+z_i)\}_{i \in [n]}$} 
    ($(P2.west)+(-5em, -9em)$);  


\node[textnode, name = result, below of = P2, yshift=-9em] 
    {$\alpha_i \leftarrow \text{PEPRF}.\mathsf{PubEval}(pk, z_i, w_i)$\\$y_i \leftarrow v_i - \alpha_i$}; 
\end{tikzpicture}
\end{center}

\caption{OPRF from input-homomorphic PEPRF}\label{figure:OPRF-from-PEPRF}
\end{figure} 


\begin{trivlist}
\item \textbf{Correctness.} The correctness of the above OPRF construction follows from that of 
    input-homomorphic PEPRF.   
\end{trivlist}

\begin{theorem}
The OPRF protocol described in Figure~\ref{figure:OPRF-from-PEPRF} 
is secure in the semi-honest model assuming $\mathsf{H}$ is a random oracle and $F$ is a family of PEPRFs.
\end{theorem}

\begin{proof}
In the above construction $F_{sk}(x): SK \times \{0,1\}^\ell \rightarrow Y$ is defined as $G_{sk}(\mathsf{H}(x))$. 
By assuming $\mathsf{H}$ is a random oracle, 
we can reduce the pseudorandomness of $F$ to the weak pseudorandomness of $G$. 
In other words, $\mathsf{H}$ amplifies weak pseudorandomness to standard pseudorandomness 
by leveraging \emph{programmability}. 

We then prove the above construction is indeed a secure OPRF protocol. 
We exhibit simulators $\mathsf{Sim}_{P_1}$ and $\mathsf{Sim}_{P_2}$ for simulating corrupt $P_1$ and $P_2$ respectively, 
and argue the indistinguishability of the simulated transcript from the real execution. 

\begin{trivlist}
\item \textbf{Security against corrupt server.} $\mathsf{Sim}_{P_1}$ simulates the view of corrupt server $P_1$, 
    which consists of $P_1$'s randomness, input, output and received messages.
    We formally show $\mathsf{Sim}_{P_1}$'s simulation is indistinguishable from the real execution 
    via a sequence of hybrid transcripts. 

\item $\text{Hybrid}_0$: $P_1$'s view in the real protocol.

\item $\text{Hybrid}_1$: Given $P_1$'s input $X$, $\mathsf{Sim}_{P_1}$ 
    chooses the randomness for $P_1$ (i.e., picks $r_i \sample R$ to sample $(z_i, w_i) \in \mathsf{R}_L$), 
    and simulates with the knowledge of $X$. 
    $\mathsf{Sim}_{P_1}$ outputs $(\mathsf{H}(x_i)+z_i, \dots, \mathsf{H}(x_n)+z_n)$.      
    Clearly, $\mathsf{Sim}_{P_1}$'s simulated view in $\text{Hybrid}_1$ is identical to $P_1$'s real view. 

\item $\text{Hybrid}_2$: $\mathsf{Sim}_{P_1}$ simulates without the knowledge of $X$. 
    $\mathsf{Sim}_{P_1}$ picks $r_i \sample R$ to sample $(z_i, w_i) \in \mathsf{R}_L$, 
    then outputs $(z_1, \dots, z_n)$.     

Clearly, the simulated view in $\text{Hybrid}_1$ and $\text{Hybrid}_2$ are identical due to $z_i$ 
distributes uniformly at random over $L$. 
\end{trivlist}

\begin{trivlist}
\item \textbf{Security against corrupt client.} $\mathsf{Sim}_{P_2}$ simulates the view of corrupt client $P_2$, 
    which consists of $P_2$'s randomness, input, output and received messages.
    We formally show $\mathsf{Sim}_{P_2}$'s simulation is indistinguishable from the real execution via a sequence 
    of hybrid transcripts. 


\item $\text{Hybrid}_0$: $P_2$'s view in the real protocol.

\item $\text{Hybrid}_1$: Given $P_2$'s input $x_i$, randomness $r_i$, 
    and output $pk$ and $y_i$, 
    $\mathsf{Sim}_{P_2}$ samples $(z_i, w_i) \in \mathsf{R}_L$, 
    then simulates the received messages $v_i:= y_i + \text{PEPRF}.\mathsf{PubEval}(pk, z_i, w_i)$. 

Clearly, the simulated view in $\text{Hybrid}_1$ and $\text{Hybrid}_2$ are identical 
due to the publicly evaluable property of the underlying PEPRF. 
\end{trivlist}
This proves the theorem. 
\end{proof}

\section{Instantiations of Input-Homomorphic PEPRF}
It remains to give efficient instantiations of input-homomorphic PEPRF.

\subsection{Input-Homomorphic PEPRF from the DDH Assumption}\label{subsec:IHPEPRF-from-DDH}
\begin{itemize}
\item $\mathsf{Setup}(1^\kappa)$: on input a security parameter $\kappa$, 
    runs $(\mathbb{G}, g, p) \leftarrow \mathsf{GroupGen}(1^\kappa)$, 
    output public parameter $pp$ specifying $(F, PK, SK, X, \allowbreak Y, L, W)$, 
    where $PK = \mathbb{G}$, $SK = \mathbb{Z}_p$, $X = L = Y = \mathbb{G}$, $W = \mathbb{Z}_p$, 
    $F: SK \times X \rightarrow Y$ is defined as $F_{sk}(x):=x^{sk}$, 
    $(x, w) \in \mathsf{R}_L$ iff $g^w = x$.   

\item $\mathsf{KeyGen}(pp)$: on input $pp$, picks secret key $sk \sample \mathbb{Z}_p$ 
        and computes public key $pk \leftarrow g^{sk}$. 
    
\item $\mathsf{PrivEval}(sk, x)$: on input $sk$ and $x \in X$, output $y \leftarrow x^{sk}$.  

\item $\mathsf{PubEval}(pk, x, w)$: on input $pk$ and $x \in L$ together with a witness $w \in W$, 
        output $y \leftarrow pk^w$.  
\end{itemize}

\begin{trivlist}
\item \textbf{Input homomorphic.} $\forall x_1, x_2 \in X$, we have $(x_1 \cdot x_2)^{sk} = x_1^{sk} \cdot x_2^{sk}$. 
\end{trivlist}

\begin{theorem}
The above construction is weakly pseudorandom if the DDH assumption holds. 
\end{theorem}

\begin{proof}
The weak pseudorandomness immediately follows from the DDH assumption. 
We omit the security proof here due to its straightforwardness. 
\end{proof}

The above construction and security proof directly carry to the weak-pusudorandom EGA setting.

\subsection{Input-Homomorphic PEPRF from the LWE Assumption}\label{subsec:IHPEPRF-from-LWE}
LWE is parameterized by positive integers $n$ and $q$, and an error distribution $\chi$ over $\mathbb{Z}$. 
For concreteness, $n$ and $q$ can be thought of as roughly the same as in SIS, 
and $\chi$ is usually taken to be a discrete Gaussian of width $\alpha q$ for some $\alpha < 1$, 
which is often called the relative ``error rate''. 


\begin{definition}[LWE distribution] 
For a vector $s \in \mathbb{Z}_q^n$ called the secret, the LWE distribution $A_{s,\chi}$ 
over $\mathbb{Z}_q^n \times \mathbb{Z}_q$ is sampled by choosing $a \in \mathbb{Z}_q^n$ uniformly at random, 
choosing $e \sample \chi$, and outputting $(a, b = \langle s, a\rangle + e \bmod q)$.
\end{definition}

\begin{definition}[Decision-LWE$_{n,q,\chi,m}$]
Given $m$ independent samples $(a_i, b_i) \in \mathbb{Z}_q^n \times \mathbb{Z}_q$ where every sample is distributed 
according to either: (1) $A_{s,\chi}$ for a uniformly random $s \in \mathbb{Z}_q$ (fixed for all samples), 
or (2) the uniform distribution, distinguish which is the case (with non-negligible advantage).
\end{definition}


We give a LWE-based PEPRF as below. 
\begin{itemize}
\item $\mathsf{Setup}(1^\kappa)$: on input a security parameter $\kappa$, first chooses a uniformly random 
    $\mathbf{A} \in \mathbb{Z}_q^{n \times m}$, 
    output public parameter $pp$ specifying $(F, PK, SK, X, \allowbreak Y, L, W)$, 
    where $PK = \mathbb{Z}_q^m$, $SK = \mathbb{Z}_q^n$, $X = L = \mathbb{Z}_q^{n+1}$, $W = \{0,1\}^{m+1}$. 
    Here the collection of $\mathcal{NP}$ languages indexed by $pk = \mathbf{u} \in \mathbb{Z}_q^m$ is defined as:
    \begin{equation*}
        L_\mathbf{u} = \{\mathbf{x} = (\mathbf{c}, v) \in \mathbb{Z}_q^{n+1} | \exists 
    \mathbf{w} = (\mathbf{a} \in \{0,1\}^m, b \in \{0,1\}) \text{~s.t.~} \mathbf{c} = \mathbf{A} \cdot \mathbf{a}, 
    v = \mathbf{u}^t \cdot \mathbf{a} + b \cdot \lfloor q/2 \rceil\}
    \end{equation*}   

    $F: SK \times X \rightarrow Y$ is defined as 
    $F_{\mathbf{s}}(\mathbf{x} = (\mathbf{c}, v)):= \mathsf{Decode}(v - \mathbf{s}^t \cdot \mathbf{c})$, 
    where $\mathsf{Decode}(z)$ outputs ``0'' if $z$ is closer to $0$ or ``1'' if $z$ is closer to $\lfloor q/2 \rceil$ 
    modulo $q$.  

\item $\mathsf{KeyGen}(pp)$: on input $pp$, first picks a secret key $\mathbf{s} \sample \mathbb{Z}_q^n$, 
    then picks $\mathbf{e} \sample \chi^m$ 
    and computes public key $\mathbf{u} \leftarrow \mathbf{A}^t \cdot \mathbf{s} + \mathbf{e} \in \mathbb{Z}_q^m$. 
    
\item $\mathsf{PrivEval}(\mathbf{s}, \mathbf{x})$: on input $\mathbf{s}$ and $\mathbf{x} = (\mathbf{c}, v)$, 
    output $y \leftarrow \mathsf{Decode}(v - \mathbf{s}^t \cdot \mathbf{c}) \in \{0,1\}$.  

\item $\mathsf{PubEval}(\mathbf{u}, \mathbf{x}, \mathbf{w})$: on input public key $\mathbf{u}$ 
    and $\mathbf{x} = (\mathbf{u}, v) \in L$ together with a witness $\mathbf{w}=(\mathbf{a}, b) \in W$, 
    output $y \leftarrow b$.  
\end{itemize}

\begin{trivlist}
\item \textbf{Correctness.} Note that $\mathbf{s}^t \cdot \mathbf{c} - \mathbf{u}^t \cdot \mathbf{a} = 
    \mathbf{s}^t \cdot \mathbf{A} \cdot \mathbf{a} - (\mathbf{A}^t \cdot \mathbf{s}  + \mathbf{e})^t \cdot \mathbf{a} 
    = \mathbf{e}^t \cdot \mathbf{a} \in \mathbb{Z}_q$. 
    The correctness holds as long as $(\mathbf{e}^t \cdot \mathbf{a})/q$ is negligible in $\kappa$.

\item \textbf{Input homomorphic.} For any $\mathbf{x}_1 = (\mathbf{A} \cdot \mathbf{a}_1, 
    \mathbf{u}^t \cdot \mathbf{a}_1 + b_1 \cdot \lfloor q/2\rceil)$ 
    and $\mathbf{x}_2 = (\mathbf{A} \cdot \mathbf{a}_2, 
    \mathbf{u}^t \cdot \mathbf{a}_2 + b_2 \cdot \lfloor q/2\rceil)$, we have: 
\begin{eqnarray*}
    F_\mathbf{s}(\mathbf{x}_1+\mathbf{x}_2) &=& \mathsf{Decode}(v_1 - \mathbf{s}^t \cdot \mathbf{c}_1 + 
        v_2 - \mathbf{s}^t \cdot \mathbf{c}_2)\\ 
        &=& \mathsf{Decode}(\mathbf{e}^t \cdot \mathbf{a}_1 + b_1\lfloor q/2\rceil + 
        \mathbf{e}^t \cdot \mathbf{a}_2 + b_2\lfloor q/2\rceil)\\ 
        &\approx& \mathsf{Decode}((b_1+b_2)\lfloor q/2\rceil) \\
        & = & b_1+b_2 \approx F_\mathbf{s}(\mathbf{x}_1) + F_\mathbf{s}(\mathbf{x}_2)
\end{eqnarray*} 

\item \textbf{Security.} The weak pseudorandomness follows from the following hybrid arguments. 

\begin{itemize}
    \item $\text{Game}_0$: the original security game. 
        $\mathcal{A}$ gets to see $(pk = \mathbf{u}, \mathbf{x} \sample L_\mathbf{u}, k^*)$, 
        where $k^* = b$ if $\beta = 0$ and $k^*$ is a random bit if $\beta=1$.  

    \item $\text{Game}_1$: same as $\text{Game}_0$ except that $\mathcal{CH}$ changes $\mathbf{u}$ 
        from the LWE distribution to a uniform distribution over $\mathbb{Z}_q^m$.  
        By the LWE assumption, we have $\text{Game}_0 \approx_c \text{Game}_1$. 

    \item $\text{Game}_2$: same as $\text{Game}_1$ except that $\mathcal{CH}$ changes $\mathbf{x}$ 
        to a uniform distribution over $\mathbb{Z}_q^{n+1}$. 
        By the leftover hash lemma, we have $\text{Game}_1 \approx_s \text{Game}_2$. 
        Clearly, (even unbounded) $\mathcal{A}$'s advantage in $\text{Game}_2$ is zero.  
\end{itemize}

\end{trivlist}

\subsection{Input-Homomorphic PEPRF from the Ring-LWR Assumption}
Define ring $R_q=\mathbb{Z}_q[X]/(X^n+1), R_p=\mathbb{Z}_p[X]/(X^n+1)$, where $q>p\geq 2$ are integers, $n=2^k$ for integer $k>0$, and $X^n+1$ is the $2^{k+1}$-th cyclotomic polynomial.

Define the modular function $[\cdot]_p: \mathbb{Z}\rightarrow\mathbb{Z}_p$ as $[x]_p= x \mod p$, and
define the rounding function $\lfloor\cdot\rceil_p: \mathbb{Z}_q\rightarrow\mathbb{Z}_p$ as $\lfloor x\rceil_p=\lfloor\frac{p}{q}\cdot x\rceil \mod p$. The definition can be extended naturally in a coefficient-wise manner on $R_q$ and $R_p$.

\begin{itemize}
\item $\mathsf{Setup}(1^\kappa):$ define $pp=(F, PK, SK, X, Y, L, W)$, where $PK=R_p$, $SK=R_q$, $X=Y=L=R_p$, $W=R_q$. \\
    $F:SK\times X\rightarrow Y$ is defined as $F_{sk}(x)=\lfloor x\cdot sk\rceil_p$. \\
    For public parameter $a\in R_q$, $(x,w)\in R_L$ iff $x=\lfloor aw\rceil_p$.
    %The language $L=\{x\in L | \exists r\in W, s.t., x=\lfloor ar\rceil_p\}$.

\item $\mathsf{KeyGen}(pp):$ on input $pp$, select the secret key $sk\leftarrow \chi(R_q)$, where $\chi$ is a sparse distribution over $R_q$, and set the public key as $pk\leftarrow \lfloor a\cdot sk\rceil_p$.

\item $\mathsf{PrivEval}(sk,x):$ on input $sk$ and $x\in X$, output $y\leftarrow \lfloor x\cdot sk\rceil_p$.

\item $\mathsf{PubEval}(pk,x,w):$ on input $pk$ and $x\in L$ with the witness $w\in W$, output $y\leftarrow \lfloor pk\cdot w\rceil_p$.

\end{itemize}

Properties of the above ring-LWR based IHPEPRF construction:
\begin{trivlist}
\item \textbf{Correctness.} For any $\kappa \in \mathbb{N}$, any $pp\leftarrow\mathsf{Setup}(1^\kappa)$ 
    and any $(pk, sk)\leftarrow\mathsf{KeyGen}(pp)$:
\begin{itemize}
    \item $\forall x\in X$: $F_{sk}(x) = \lfloor x\cdot sk\rceil_p = \mathsf{PrivEval}(sk,x)$.
    \item $\forall x\in L$ with witness $w$:
    \[
    \mathsf{PubEval}(pk,x,w)= \lfloor pk\cdot w\rceil_p = \lfloor[\lfloor\frac{p}{q}\cdot a\cdot sk\rceil]_p\cdot w\rceil_p,
    \]
    
    \[
    F_{sk}(x)=\lfloor x\cdot sk \rceil_p =\lfloor[\lfloor\frac{p}{q}\cdot a\cdot w\rceil]_p\cdot sk\rceil_p
    \]
\end{itemize}
\end{trivlist}

% \blue{
% I do not understand the following notations:
% \begin{enumerate}
% \item $\lfloor x\cdot sk\rceil_p$: here $x \in R_p$, $sk \in R_q$, how to define the multipilication between them?

% \item What does $[\lfloor\frac{p}{q}\cdot a\cdot sk\rceil]_p$ mean?
% \end{enumerate}

% By the way, why we have to define $pk \in R_p$ rather than $pk \in R_q$?
% }

% \textbf{A bit troublesome to make them equal: $q\gg p$}

\begin{trivlist}
\item \textbf{Input homomorphic.} A bit tricky while dealing with the rounding function: it is not linear.
\[F_{sk}(x_1+x_2)=[\lfloor\frac{p}{q}(x_1+x_2)\cdot sk\rceil]_p,\]
\[F_{sk}(x_1)+F_{sk}(x_2)=[\lfloor\frac{p}{q}\cdot x_1\cdot sk\rceil]_p+[\lfloor\frac{p}{q}\cdot x_2\cdot sk\rceil]_p.\]
If $\lfloor x_1\rceil=\lfloor x_2\rceil$, then $F_{sk}(x_1+x_2)=F_{sk}(x_1)+F_{sk}(x_2)$.
\end{trivlist}

\begin{trivlist}
\item \textbf{Security.} Guaranteed by the ring LWR assumption: $(a, \lfloor a\cdot s\rceil_p)\approx(a,u)$.
\end{trivlist}

\magenta{The above construction is floppy and could be problematic.}  

% \blue{Shall we also take into account the public key in the two distribution ensembles? }

\section{Concrete Post-quantum Secure OPRF}

By plugging PEPRF from weak pseudorandom EGA in Section~\ref{subsec:IHPEPRF-from-DDH} to our generic construction, 
we obtain a OPRF with plausible post-quantum security.  

\magenta{We encounter a technical hurdle when plugging the instantiation in Section~\ref{subsec:IHPEPRF-from-LWE} to our generic construction. 
This is because the weak pseudorandomness of $F$ is tied to special distribution over $L_\mathbf{u}$, not a uniform distribution. 
The consequence is that we can not assume there exists a random oracle from $\{0,1\}^*$ to $L_\mathbf{u}$ anymore. 
One attempt is to remove $v$ from $\mathbf{x}$ and only keep $\mathbf{c}$. 
However, this somehow requires a fuzzy and homomorphic randomness extractor, which again seems quite tough.}  


% \section*{Acknowledgments}
  

\bibliographystyle{alpha}
\bibliography{\bibpath/mycrypto}


\appendix

\section{Missing Definitions}

\end{document}

